\RequirePackage{xcolor} % to use xcolor in jhep3
%\documentclass[10pt]{article}
%\documentclass[prd,preprint]{revtex4}
\documentclass[11pt]{JHEP3}
%\documentclass[published]{JHEP3}

%\usepackage{amsmath}
%\usepackage{graphicx}
%\usepackage{amssymb}
%\usepackage{latexsym}
%\usepackage{dcolumn}% Align table columns on decimal point
%\usepackage{bm}% bold math
%\usepackage{epsfig,multicol,bbm}
\usepackage{listings}% to insert codes in tex
\usepackage[framed,numbered,autolinebreaks,useliterate]{mcode} %inline code enviroment
\usepackage{multirow} %multi-line in tabular
\usepackage{float} % fix the position of table
%\usepackage{color}
%\usepackage{xcolor}
%\usepackage{epstopdf}
%\usepackage[T1]{fontenc}
%\usepackage[latin9]{inputenc}
%\usepackage{units}
%\usepackage{esint}

\setlength{\textwidth}{17cm} %
\setlength{\textheight}{25cm} %
\setlength{\topmargin}{0.5cm} %
\setlength{\oddsidemargin}{2cm} %
%multi-line in tabular
\newcommand{\tabincell}[2]{\begin{tabular}{@{}#1@{}}#2\end{tabular}}
% the style of inserted codes
%\lstset{
%    upquote=true,
%    aboveskip={1.5\baselineskip},
%    columns=fullflexible,
%    showstringspaces=false,
%    extendedchars=true,
%    showtabs=false,
%    showspaces=false,
%    showstringspaces=false,
%    identifierstyle=\ttfamily,
%    escapeinside=``,
 \lstset{language=Fortran,
%        numbers=right,
%        numberstyle= \tiny,
%        keywordstyle= \color{blue!70},
%        commentstyle=\color{red!50!green!50!blue!50},
%        frame=shadowbox,
%         frame=L,
        rulesepcolor= \color{ red!20!green!20!blue!20},
%        breaklines=true,
%        tabsize=4,
%        basicstyle=\footnotesize\ttfamily,
        escapeinside=``,
        keywordstyle=\bfseries\color{green!60!black},
        commentstyle=\itshape\color{purple},
        identifierstyle=\color{blue},
        stringstyle=\color{orange},
%        framexleftmargin=1.2em, xleftmargin=1.2em, aboveskip=1em,
        }

%%%%%%%%%%%%%%%%%%%%%%%%%%%%%%%%%%%%%%%%%%%%%%%%%%%%%%%%%%%%
%%%%%%%%%%%%%%%%%%%%%%%%%%%%%%%%%%%%%%%%%%%%%%%%%%%%%%%%%%%%

\title{Notes on a Molecular Dynamics Package MOPDYN13}

\author{Hong-Bo Zhang\\
 Center for Combustion Energy, Tsinghua University, Beijing, 10084,
 China\\
 hongbozhang@tsinghua.edu.cn}
%\author[a]{Hong-Bo Zhang,\note{Corresponding author} }
%\affiliation[a]{Center for Combustion Energy, Tsinghua University,
%Beijing, 10084, China}
%\emailAdd{hongbozhang@tsinghua.edu.cn }

%\begin{abstract}
\abstract{ MOPDYN13 is a paralleled Fortran77 package to perform ab
initio molecular dynamics (MD) simulation, empirical MD and combined
MD with the Beeman's third order integrator algorithm, in which
PM3/6 in MOPAC is used as the quantum chemistry potential in ab
initio MD. Although MOPDYN13 is powerful, its manual is not detailed
enough. Therefore, several issues on how to use this package will be
summarized in this note.}
%\end{abstract}

%%%%%%%%%%%%%%%%%%%%%%%%%%%%%%%%%%%%%%%%%%%%%%%%%%%%%%%%%%%%
%%%%%%%%%%%%%%%%%%%%%%%%%%%%%%%%%%%%%%%%%%%%%%%%%%%%%%%%%%%%
\begin{document}
\newpage
\section{Keywords in Dynput File}

Files with extension 'dynput' are the configuration file (input
file) of MOPDYN13, which should be executed as

\begin{lstlisting}[language=sh]
 mopdyn13 *.dynput
\end{lstlisting}

In the dynput file, one should specify coordinates and velocities of
atoms, as well as tens of keywords of which the detailed information
will be summarized in the following.

\subsection{Keywords}\label{sec:key}

A typical dynput file should be like this:

\lstinputlisting{example.dynput}

\begin{itemize}
\item  In the line $1$, there is a name of task, which has $40$ characters
at most and there's no space within.

\item The keywords in the line $2$ should be one of the six options,
which are
\begin{lstlisting}
  MOPAC VERSION, MOPAC 6, MOPAC 7, MOPDYN * RATES, MOPDYN * MORPHOLOGY, MOPDYN * HEATTRANSFER
\end{lstlisting}
where * stands for any word. Among them, \mcode{MOPDYN * RATES} is
most frequently used in our study on collision of molecules. The
keywords in this line will determine an important parameter
\mcode{imopcall}, which will determine how to call MOPAC and how to
set atoms. Detailed information is listed in Tab.\ref{Tab:1}. As
shown in Tab.\ref{Tab:1}, the several lines below the 2nd line are
different with different value of \mcode{imopcall}. Because the
\mcode{MOPDYN for RATE} is important for our usage, we will
summarize them in the case of \mcode{imopcall=50} only.
%\end{itemize}

%\lstinline{VMAX}
%\multirow{2}{*}{}
\begin{table}%[H]
\scriptsize
\begin{centering}
\begin{tabular}{|c|c|c|c|}
\hline 2nd line & imopcall & MopacVersion & Remarks\tabularnewline
\hline \hline \tabincell{c}{MOPCA  VERSION} & 1 & 6 or 7 &
\tabincell{c}{MopacVersion and MopacType in the 3rd line,\\
MopCommand in the 4th line. LCHARGE should be F.\\PM6 isn't allowed
and wine MOPAC\_7.2 actually \\ if MopacVersion is 6}
\tabularnewline \hline MOPAC 6 & 0 & 6 & \tabincell{c}{MopacType in
the 3rd line, MopCommand should
\\not be read from dynput file. LCHARGE should be F.\\PM6 isn't allowed
and wine MOPAC\_7.2 actually} \tabularnewline \hline MOPAC 7 & 0 & 7
& the same as above except the last sentence\tabularnewline \hline
\tabincell{c}{$\bigstar$ MOPDYN {*} RATES } & 50 & 7 &
\tabincell{c}{MopSolver in the 3rd line, MopacType in the 4th
line,\\ MopCommand in the 5th line,\\Run the given MopSolver}
\tabularnewline \hline \tabincell{c}{MOPDYN {*} MORPHOLOGY} & 100 &
7 & the same as above \& configuration of atoms \tabularnewline
\hline \tabincell{c}{MOPDYN {*} HEATTRANSFER} & 500
& 7 & \tabincell{c}{ActiveBndr begins at the 3rd line, the same \\
as above \& configuration of atoms} \tabularnewline \hline
\end{tabular}
\end{centering}
\caption{The second line in dynput file\label{Tab:1}}
\end{table}

%\begin{itemize}
\item The MopacSolver is located in the third line. In this example,
\mcode{/home/youxq/.../MOPAC2012.exe mopac} will be executed.
Nevertheless, in principle, MopacSolver can be any command which can
be executed in Shell.

\item In the fourth line, MopacType is next to
\mcode{PARAMETERIZATION}, which can be \mcode{PM3,PM6,
AM1},$\ldots$. For our demands, \mcode{PM6} will be used.

\item In the fifth line, there is MopacCommands, which will be
passed to MOPAC2012.exe together with some additional paramters such
as \mcode{denout xyz gradients dcart geo-ok}. In this example, the
parameters passed to MOPAC2012.exe should be \mcode{PM6 uhf bonds
1SCF SCFCRT=1.D-10 AUX(PRECISION=9) denout (oldens) xyz gradients
dcart geo-ok}. The meaning of the parameters can be found in MOPAC's
manual, and I also list them in Sec.\ref{sec:mopac} briefly.

\item In the sixth line, there are five keys, which are \mcode{IPOT,LENER,LMOMEN,LXROT,LBOUND
} in order. \mcode{IPOT} specifies the potential between atoms. At
present, there are 13 potentials including empirical potentials,
QC(semi-empirical) potentials and combined potentials which can be
found in \mcode{setup.f}, \lstinputlisting{setup.f} where
\mcode{ICOMPOT=.T.} means that potential is a combined one. At first
glance, one may think \mcode{IPOT=3} is suitable for us to do ab
initio MD simulation, but it doesn't work since there is a bug in
\mcode{accel.f} which is discussed in detail in Sec.\ref{sec:ipot3}.
As a result, \mcode{IPOT=11} should be used for our purpose,
although in this case there will be some useless calculations which
makes computation time longer. \mcode{LENER=T} means maintaining the
total energy constant by keeping the potential energy unchanged and
rescaling the velocity every \mcode{DT} $fs$ in the case of
potential energy is smaller than the initial total energy (initial
kinetic energy plus initial potential energy). The case that the
potential energy is larger than the initial total energy will happen
when the step length in the iteration is too long and two atoms are
so closed to each other. In this case, the two atoms may move even
closer and the repulsion force between them will dominate. As a
result, the potential energy will increase too much to become larger
than the initial total energy which is physical unreasonable, and
the program will interrupt. For thermalization, we should set
\mcode{LENER=F}, but for collision, we should set \mcode{LENER=T}.
If the option \mcode{LMOMEN=T}, the momentum of center of mass will
be removed every \mcode{DT} $fs$ without rescaling velocity to
maintain the original temperature and the resulting kinetic enenrgy
\mcode{ZKE} will be returned. The function of \mcode{LXORT=T} is to
remove the global angular momentum of the system every \mcode{DT}
$fs$. As the same as \mcode{LMOMEN}, \mcode{LXORT} doesn't rescale
the velocity to maintain the original temperature, neither. But
\mcode{LXORT} will not return the resulting kinetic energy. The
\mcode{subroutine periodic} to set minimum image boundary conditions
will be called if \mcode{LBOUND=T} which is useful in lattice
simulation which is not our interest, so it should be set to
\mcode{F}. Furthermore if \mcode{LBOUND=T}, \mcode{SX,SY,SZ} should
be assigned values.

\item In the seventh line, there are four options, which are
\mcode{LTEMP,LTEST,RIMAGE,LCHARG}. Among them, \mcode{LTEMP=T} is
the option to call \mcode{subroutine ADHOC} which is used to rescale
velocity to a certain temperature \mcode{TREF} (thermalization
temperature) every \mcode{DT*NTIME} in a very crude and obsolete
way. One can see Sec.\ref{sec:tem} for more detail of
\mcode{subroutine ADHOC}. In our work, \mcode{LTEMP} should be set
to \mcode{F}. The only use of \mcode{LTEST} is to determine whether
calculate the acceleration and the velocity of image/ghost atoms for
debugging via code \mcode{IF (LTEST) NLIM = NKEEP}, and
\mcode{LTEST} is different from \mcode{DEBUGM}, which will result in
outputting debug information in \mcode{*.outpt} if \mcode{DEBUGM=T}.
Therefore, \mcode{LTEST} should be set to \mcode{F}. \mcode{RIMAGE}
is not used any more, and its value will not pass to any buffer.
Therefore, it doesn't matter at all to set it \mcode{T} or
\mcode{F}. For convenience, we will always use \mcode{RIMAGE=F} in
simulation. The last option \mcode{LCHARGE=T} will output bond
orders and charges in files \mcode{*.bond,
*.charge}, and it is not supported for the MOPAC version earlier than
7. Of course, if one want to get information on bond order and
charge, keyword \mcode{bonds} should be included in MopacCommands in
line five.

\item In the eighth line, four options
\mcode{LNOISE,LTRAMP,LQUENCH,LTSTAT} stand in a row. If
\mcode{LNOISE=T}, random velocities (noise) ranging from
\mcode{-ZNOISE} to \mcode{+ZNOISE} will be added to original
velocities, followed by removing the center of mass momentum and
global angular momentum and rescaling the velocities to maintain
original temperature. The function of \mcode{LTRAMP} is to change
thermalization temperature \mcode{TREF} by \mcode{TRAMP} every
\mcode{DT} fetoseconds. It may be useful in heating or cooling.
Keyword \mcode{LQUENCH} is used to find minimal potential energy
structure. If atom $i$ is climbing up the potential, its velocity at
time $t+DT$ will be smaller than that at time $t$, i.e.
$|v_i(t+DT)|<|v_i(t)|$. At this moment, we set $\vec{v}_i(t+DT)$ to
zero if \mcode{LQUENCH=T}. Obviously, at next time $t+2DT$, the atom
$i$ will climb down the potential since its velocity
$\vec{v}_i(t+DT)=0$. Obviously, it will approach the bottom of the
potential if we keep doing so. Therefore, it will useful to find
stable structure of a molecule. \mcode{LTSTAT} is an advanced
version of \mcode{LTEMP}. \mcode{Subroutine thermstat} will be
called to control the temperature of atoms with z-coordinate smaller
than \mcode{ZTHERM} to maintain \mcode{TREF} if \mcode{LTSTAT=T},
and the temperature will approach \mcode{TREF} in \mcode{TCONST*DT}
fetoseconds in the condition that the potential energy is constant.
However, in realistic case, the energy will transfer between
potential energy and kinetic energy, so \mcode{TCONST*DT}
fetoseconds is a rough estimation. Moreover, \mcode{subroutine
thermstat} is called every \mcode{DT} fetoseconds. There are three
related parameters in dynput file if \mcode{LTSTAT=T}. They are
\mcode{TREF,TCONST,ZTHERM}. Noted that \mcode{LTEMP} will override
\mcode{LTSTAT}, so if one wants to do thermalization, keywords
should be set to be \mcode{LTEMP=F, LTSTAT=T,LENER=F}. For more
information on temperature control, see Sec.\ref{sec:tem}.

\item In the ninth line, there are three parameters, \mcode{N,NIMAGE,NRECT
}. \mcode{N} is the number of atoms you input dynput file.
\mcode{NIMAGE} is the number of image/ghost atoms, which either the
empirical atoms near the junction of quantum atoms region and
empirical atoms region in the combined potential MD or the atoms
near the boundary in periodic boundary condition. \mcode{NRECT} is
the number of H atoms added to make surface C or Si satisfy sp3
hybridization. (I am not sure with these, I don't understand the
code relevant to \mcode{NIMAGE}.) In our study, we should set
\mcode{NIMAGE=0,NRECT=0}.

\item In the tenth line, there are five parameters,
\mcode{ICF,VMAX,ZNOISE,TREF,IDENRC}. \mcode{ICF} is the flag of
initial condition. If \mcode{ICF=1}, there is only initial position
in dynput file, and the initial velocities will be generated
randomly, ranging from \mcode{-VMAX} to \mcode{VMAX} without
removing center mass momentum and global angular momentum. If
\mcode{ICF=2}, both initial position and initial velocity will be
read from dynput file. If \mcode{ICF=3}, it is said that molecular
orbital coefficients or density matrices will be read, but I didn't
find any relevant codes on what will happen if \mcode{ICF=3}.
Indeed, in file \mcode{SIZE}, there are some information on the
orbitals, but there is nothing relevant of these orbital information
in the Fortran or C files. Maybe, it is still in developing. For
thermalization, one needs to set \mcode{ICF=1}; for collision, one
needs to set \mcode{ICF=2}. \mcode{VMAX, ZNOISE, TREF} are in unit
of $\AA/fs,\AA/fs,K$,respectively, and have been explained before.
Noted that \mcode{TREF} is relevant to three options, they are
\mcode{LTEMP,LTSTAT,LTRAMP}. \mcode{IDENRC} is useless and doesn't
pass to any buffer. Maybe, it is in developing.

\item In the eleventh line, there are three parameters, they are
\mcode{DT,NTIME,NSNAP}. \mcode{DT} is the incremental timestep of MD
simulation in unit of $fs$; \mcode{NTIME} is the number of time
steps between outputs; and \mcode{NSNAP} is the number of output
calls (snapshots of the system).

\item In the twelfth line, there are three parameters,
\mcode{SX,SY,SZ}, which is the boundary of the system or lattice.
They are useful only if \mcode{LBOUND=T}. In our work, they are
useless and could be set to zero.

\item In the thirteenth line, there are one parameter and one
option, \mcode{TRAMP,RTSTAT}. \mcode{TRAMP} in unit of $K$ has been
explained before. And \mcode{RTSTAT} is the same as \mcode{LTSTAT}
except for rescaling angular velocity. They share the same
\mcode{TREF} and \mcode{TCONST}, but \mcode{RTSTAT} has nothing to
do with \mcode{ZTHERM}.

\item In the fourteenth line, there are two parameters, \mcode{TCONST,ZTHERM
}, which have been introduced before. \mcode{TCONST} should be an
integer, and it means the number of time step \mcode{DT}. In our
work, \mcode{ZTHERM} in $\AA$ should be set to a large number.

\item In the fifteenth line, there are two parameters, they are
\mcode{SCALE,DBOUND}, which should not appear in the dynput file
except satisfying the condition:\\
%\begin{lstlisting}
\mcode{IF(IMOPCALL.GE.50.AND.(IPOT.EQ.11.OR.IPOT.EQ.15))}\\
%\end{lstlisting}
which is case we met in MD simulation. \mcode{DBOUND}
will be used in \mcode{empni.f} as cutoff, and \mcode{SCALE} will be
also used in \mcode{empni.f} as scale factor which value will be
re-assigned in \mcode{empni.f}. Since we will perform ab initio MD
instead of using empirical potential \mcode{empni}, these two
parameter are not of interest.

\item Beginning from the sixteenth line, they are atom position
specification. There are atomic cartesian coordinates in $\AA$
followed by atomic number, number of potential and value of
\mcode{JFIX=O} (non fixed atom) or \mcode{JFIX=1} (fixed atom). In
our case, number of potential \mcode{ICOL} and \mcode{JFIX} should
be assigned to 3 and and 0, respectively.

\item Followed atom position, there should be atom velocity
specification if \mcode{ICF=2}. Velocity is in unit of $\AA/fs$.

\item At the end of the dynput file, there may be a list of
\mcode{IGHOST \& ILINK} if \mcode{imopcall=500} for heat transfer
purpose. I haven't understood the detail of heat transfer
calculation. It seems that it calculate the particle number flux and
momentum flux in lattice bounded by \mcode{SX,SY,SZ} with periodic
boundary condition by gas tooth module. However, it is irrelevant to
our work.

\end{itemize}


\subsection{Examples}

\begin{itemize}
\item Thermalization, isothermal: LENER = F, LTSTAT = T
\begin{lstlisting}
r51m_thermal_5ps_1600K                                        MOPDYN
FOR RATES                                                  RUN
/home/youxq/you/mopac/MOPAC2012.exe              PARAMETERIZATION
PM6                                              uhf bonds 1SCF
SCFCRT=1.D-10 AUX(PRECISION=9)
    11  F   T   T   F                   ! IPOT,LENER,LMOMEN,LXROT,LBOUND
        F   F   F   F                   ! LTEMP,LTEST,RIMAGE,LCHARG
        F   F   F   T                   ! LNOISE,LTRAMP,LQUENCH,LTSTAT
     36  0    0                         ! N,NIMAGE,NRECT
    1    .00001   .00000   1600.0   1   ! ICF,VMAX,ZNOISE,TREF,IDENRC
    0.5     5       2000                ! DT,NTIME,NSNAP
   0.0 0.0 0.0                          ! SX,SY,SZ
     .000000         F                  ! TRAMP,RTSTAT
        1000.00     100.000000             ! TCONST,ZTHERM
        0.544  20.0                      ! SCALE DBOUND
                  6.11922000   -0.72470900    0.00000200        6 3 0
                  4.95450000   -1.41661300    0.00000200        6 3 0
                  3.67104300   -0.72752800    0.00000000        6 3 0
                  ................
\end{lstlisting}

\item Collision, adiabatic: LENER = T, LTSTAT = F
\begin{lstlisting}
r51d_collision_z_5ps_1600K
MOPDYN FOR RATES
RUN /home/youxq/you/mopac/MOPAC2012.exe
PARAMETERIZATION PM6
uhf bonds 1SCF SCFCRT=1.D-10 AUX(PRECISION=9)
    11  T   T   T   F                   ! IPOT,LENER,LMOMEN,LXROT,LBOUND
        F   F   F   T                   ! LTEMP,LTEST,RIMAGE,LCHARG
        F   F   F   F                   ! LNOISE,LTRAMP,LQUENCH,LTSTAT
     72  0    0                         ! N,NIMAGE,NRECT
    2    .00001   .00000   1600.0   1   ! ICF,VMAX,ZNOISE,TREF,IDENRC
    0.5     5       2000                ! DT,NTIME,NSNAP
   0.0 0.0 0.0                          ! SX,SY,SZ
     .000000         F                  ! TRAMP,RTSTAT
        1000.00     100.000000             ! TCONST,ZTHERM
        0.544  20.0                      ! SCALE DBOUND
        6.11922000   -0.72470900    0.00000200     6    3    0
        ...............
        3.67104300   -0.72752800    4.00000000     6    3    0
        ................
        -0.016433    0.000293   0.007034
        ................
        0.010590    0.007662   -0.022977
\end{lstlisting}

\item Quenching: LENER = F, LTSTAT = F, LQUENCH = T
\begin{lstlisting}
r51m_quench_5ps_1600K
MOPDYN FOR RATES
RUN /home/youxq/you/mopac/MOPAC2012.exe
PARAMETERIZATION PM6
uhf bonds 1SCF SCFCRT=1.D-10 AUX(PRECISION=9)
    11  F   T   T   F                   ! IPOT,LENER,LMOMEN,LXROT,LBOUND
        F   F   F   F                   ! LTEMP,LTEST,RIMAGE,LCHARG
        F   F   T   F                   ! LNOISE,LTRAMP,LQUENCH,LTSTAT
     36  0    0                         ! N,NIMAGE,NRECT
    2    .00001   .00000   1600.0   1   ! ICF,VMAX,ZNOISE,TREF,IDENRC
    0.5     5       2000                ! DT,NTIME,NSNAP
   0.0 0.0 0.0                          ! SX,SY,SZ
     .000000         F                  ! TRAMP,RTSTAT
        1000.00     100.000000             ! TCONST,ZTHERM
        0.544  20.0                      ! SCALE DBOUND
                  6.11922000   -0.72470900    0.00000200        6 3 0
                  3.67104300   -0.72752800    0.00000000        6 3 0
                  ...............
                  -0.016433    0.000293   0.007034
                  0.010590    0.007662   -0.019977
                  ...............
\end{lstlisting}

\item Relaxation: LENER = T, LTSTAT = F 
\begin{lstlisting}
r51m_relax_5ps_1600K                                          MOPDYN
FOR RATES                                                  RUN
/home/youxq/you/mopac/MOPAC2012.exe              PARAMETERIZATION
PM6                                              uhf bonds 1SCF
SCFCRT=1.D-10 AUX(PRECISION=9)
    11  T   T   T   F                   ! IPOT,LENER,LMOMEN,LXROT,LBOUND
        F   F   F   F                   ! LTEMP,LTEST,RIMAGE,LCHARG
        F   F   F   F                   ! LNOISE,LTRAMP,LQUENCH,LTSTAT
     36  0    0                         ! N,NIMAGE,NRECT
    2    .00001   .00000   1600.0   1   ! ICF,VMAX,ZNOISE,TREF,IDENRC
    0.5     5       2000                ! DT,NTIME,NSNAP
   0.0 0.0 0.0                          ! SX,SY,SZ
     .000000         F                  ! TRAMP,RTSTAT
        1000.00     100.000000             ! TCONST,ZTHERM
        0.544  20.0                      ! SCALE DBOUND
                  6.11922000   -0.72470900    0.00000200        6 3 0
                  3.67104300   -0.72752800    0.00000000        6 3 0
                  ...............
                  -0.016433    0.000293   0.007034
                  0.010590    0.007662   -0.019977
                  ...............
\end{lstlisting}
After thermalization, there should be a period of relaxation, which 
is adiabtic and no rescaling velocities, in order to ensure 
adiabatic equilibrium at 1600 K had been achieved.
\end{itemize}

%%%%%%%%%%%%%%%%%%%%%%%%%%%%%%%%%%%%%%%%%%%%%%%%%%%%%%%%%%%%%%%%%%%%%%%%%%
%%%%%%%%%%%%%%%%%%%%%%%%%%%%%%%%%%%%%%%%%%%%%%%%%%%%%%%%%%%%%%%%%%%%%%%%%%

\section{Output Files}
There are many output files generated in MD simulation.

\subsection{Important Output Files}
\begin{itemize}
\item \mcode{*.itinf}. Output \mcode{ZKE,ZPOT,TOTEN,TEMP} in Hartrees, Hartrees, Hartrees and K, respectively every
\mcode{NTIME*DT} fs.

\item \mcode{*.q3d}. Output coordinates in Chem3D format which can be used to produce snaps or
movie. After
\begin{lstlisting}
sed -in 's/N=//g' *.q3d                                            
mv *.q3d *.xyz
\end{lstlisting}
, JMOL can be used to make snaps or movies. Generated if 
\mcode{IPOT=7,10,11,15}.

\item \mcode{*.coord}. Record the coordinate of atoms with
\mcode{ICOL=3} every \mcode{NTIME*DT} fs. Generated if
\mcode{IPOT=7,10,11,15}.

\item \mcode{*.velocity}. Record the velocities of atoms with
\mcode{ICOL=3} every \mcode{NTIME*DT} fs. Generated if
\mcode{IPOT=7,10,11,15}.

\item \mcode{*.accel}. Record the accelerations of atoms with
\mcode{ICOL=3} every \mcode{NTIME*DT} fs. Generated if
\mcode{IPOT=7,10,11,15}.

\item \mcode{*.bond}. Bond order matrix. Generated every \mcode{NTIME*DT} fs if
\mcode{LCHARG=T}

\item \mcode{*.1coord}. Record the coordinates of atoms with \mcode{ICOL=3}
every \mcode{DT} fetoseconds.

\item \mcode{*.rstart}. Record the coordinates, velocities and
ghost/links of the last snap.
\end{itemize}

\subsection{Not So Important Files}
\begin{itemize}
\item \mcode{*.charge}. Charge and dipole, without orbital. Generated every \mcode{NTIME*DT} fs if \mcode{LCHARG=T}
populations.

\item \mcode{*.homo}. SOMO and LUMO for $\alpha$ and $\beta$ orbitals. Generated every
\mcode{NTIME*DT} fs.

\item \mcode{*.outpt}. Some debug information,such as MopacVersion,
iteration number, NQC, potential information and so on.

\item \mcode{*.infout}. To check whether dynput file has been read correctly.
Almost the same as *.dynput except that \mcode{ICF -> ICF2}, there
is velocity specification if  \mcode{ICF=1}, 'momentum and angular
momentum are not removed' and no IGHOST/ILINK part.

\item \mcode{mopac.*}. File mopac.dat is the input file of MOPAC,
while mopac.out and mopac.aux are output files of MOPAC. They will
be updated every \mcode{DT} fetoseconds.

\item \mcode{mindist.dat}. Obsolete.

\item \mcode{*.checkthedata}. I don't know what is the file for.

\item \mcode{*.links}. Results from \mcode{ghost.f}. The nearest \mcode{NIMAGE} empirical atoms to the QC
region.
\end{itemize}

%%%%%%%%%%%%%%%%%%%%%%%%%%%%%%%%%%%%%%%%%%%%%%%%%%%%%%%%%%%%%%%%%%%%%
%%%%%%%%%%%%%%%%%%%%%%%%%%%%%%%%%%%%%%%%%%%%%%%%%%%%%%%%%%%%%%%%%%%%%

\section{Bugs}
\subsection{Compile}
The MOPDYN13 package is written in both Fortran and C, with Fortran
as the main part and C as the driver to call main program and
MOPAC2012.exe. As a result, one has to mixed compile Fortran and C.
As everyone known, the names of variables, functions and subroutines
in compiled object files (\mcode{*.o}) are exactly the same as the
ones in the C source files (\mcode{*.c}). On the contrary, the names
in \mcode{*.o} are the ones in the Fortran source files
(\mcode{*.f}) followed by '\mcode{_}'. Therefore, when one tries to
link the object files compiled from Fortran and C, the functions or
subroutines will be not found when Fortran tries to call C or  vise
versa.

To manage this problem, the following flag can be added to compile
Fortran source files.
\begin{lstlisting}
-fno-underscoring
\end{lstlisting}
The function of this flag is to remove the '\mcode{_}' in object
files compiled from Fortran source file.

In summary, the compiling command in \mcode{Makefile} should be
\begin{lstlisting}
f77  -c  -g -O2 -fno-underscoring *.f
gcc -c *.c
f77 -o *.o
\end{lstlisting}
where \mcode{f77} cannot be changed to \mcode{gfortran} since the
code is written in Fortran77, in which some functions and
subroutines are not available in Fortran90.

If one wants to include \mcode{*.f90} in mixed compiling with C and
Fortran77, the flag
\begin{lstlisting}
-lgfortran -lgfortranbegin
\end{lstlisting}
should be added to link options.


\subsection{IPOT=3}\label{sec:ipot3}
If \mcode{IPOT} is assigned a value $3$, MOPDYN13 will be failed to
call MOPAC2012.exe since there is not any atom specification in
\mcode{mopac.dat}. In fact, if \mcode{IPOT=3}, \mcode{subroutine
callmopac(NQC,XQC,YQC,ZQC...)} will be called before assigning
\mcode{NQC,XQC,YQC,ZQC...} any value. As a result, \mcode{NQC=0} and
no atom information will written into \mcode{mopac.dat}.

However, we can overcome this bug by two ways, one is to modify
\mcode{callmopac(NQC,XQC...)} to \mcode{callmopac(N,X...)}, the
other is to assign \mcode{IPOT=11} and set the number of potential
after atoms cartesian coordinates \mcode{ICOL=3}. We will adopt the
second way.

In the second way, firstly the empirical potential energy
\mcode{ZPOT} in model \mcode{empni} will be calculated and the
empirical force between (quantum) atoms with \mcode{ICOL=3} will be
assigned to zero. Secondly, after assigning
\mcode{NQC,XQC,YQC,ZQC,IQTYP...}, \mcode{subroutine
callmopac(NQC,XQC,YQC,ZQC...)} will be called to get quantum force
\mcode{FMOP} and the quantum potential energy \mcode{ZPOT} which
will override the empirical one. Finally, the quantum potential and
force \mcode{ZPOT} and \mcode{FMOP} will be used in MD simulation.
Therefore, in this method, everything works well, except some
computation time is wasted on calling \mcode{subroutine empni} in
every iteration.

\subsection{q3d File}
File \mcode{*.q3d} restores the coordinate information which can be
used to generate pictures and movies by software such as JMOL. The
first line of \mcode{*.q3d} is not correct, which should be 'TIME= 0
fs' instead of 'TIME = ? fs' where '?' represent some unfortunate
number. Because in the first time to call \mcode{subroutine prt3d},
the buffers \mcode{ITIME} and \mcode{ISNAP} haven't been assigned
any value.
\subsection{itinf File}
There should be a title line
\begin{lstlisting}
Kinetic energy (hartrees), Potential energy (hartrees), Total Energy
(hartrees), Temperature (K), Time (fs).
\end{lstlisting}

%%%%%%%%%%%%%%%%%%%%%%%%%%%%%%%%%%%%%%%%%%%%%%%%%%%%%%%%%%%%%%%%%%%%%
%%%%%%%%%%%%%%%%%%%%%%%%%%%%%%%%%%%%%%%%%%%%%%%%%%%%%%%%%%%%%%%%%%%%%

\section{Algorithm}
\subsection{Verlet}
In MD, Verlet algorithm is frequently used to solve Newtonian
equation with initial condition $x_0$ and $v_0$.

From the definition of acceleration and using forward difference in
velocity, backward difference in acceleration, we have
\[ \frac{\frac{x_{n+1}-x_n}{\Delta
t}-\frac{x_{n}-x_{n-1}}{\Delta t}}{\Delta t} = a_n, \] after
re-organizing the terms, one gets
\[  x_{n+1} = 2 x_n -x_{n-1} + a_n \Delta t^2 + \mathcal{O}(\Delta t^4) \]
where there's no velocity term. It tells us $x_{n+1}$ is readily
solved once $x_n$ and $x_{n-1}$ are known with an error of order
$\Delta t^4$. However, it is obviously that $x_0$ and $x_1$ must be
given if one wants to obtain $x_n$ for arbitrary $n$ iteratively.
Although $x_0$ is already known as initial condition, $x_1$ is
unknown. Therefore, solving $x_{n}$ turns into a problem of getting
$x_1$. In Verlet method, $x_1$ is given by the familiar relation
\[ x_1=x_0+v_0 \Delta t + \frac{1}{2} a_0 \Delta t^2 \] with a low
degree of accuracy which will propagate into $x_n$. At present, all
the $x_n$ for $\forall n$ have been determined. In order to close
the problem, the velocity $v_n$ should be solved, too. Velocity
Verlet reads
\[ v(t+\Delta t) = v(t) + \frac{a(t)+a(t+\Delta t)}{2}\Delta t. \]

In summary, there are three iterative relations in this algorithm,
\begin{eqnarray}
x_1\quad &=& x_0+v_0 \Delta t + \frac{1}{2} a_0 \Delta t^2
\label{eq:verlet1} \\ x_{n+1} &=& 2x_n -x_{n-1} + a_n \Delta t^2
\label{eq:verlet2}\\  v_{n+1} &=& v_n + \frac{a_n+a_{n+1}}{2}\Delta
t \label{eq:verlet3}
\end{eqnarray}
 Therefore, in Verlet algorithm, one should obtain $a_0=F_0/m$
 where $F_0=F(x_0)$ is the force law such as gradient of LJ potential or QM potential by calling MOPAC2012.exe, followed
by $x_1$ from Eq.\ref{eq:verlet1}, $a_1=F_1/m$ from the force law,
$x_2$ from Eq.\ref{eq:verlet2}, $a_2=F_2/m$ from force law again and
$v_2$ from Eq.\ref{eq:verlet3} and so forth.

In MOPDYN13, Verlet algorithm is used to obtain position, velocity
and acceleration at time \mcode{DT} only.

\subsection{Beeman}
Beeman's algorithm is widely used in MD to solve Newtonian equation
with initial condition $x_0$ and $v_0$. In this method, variables of
$t+\Delta t/2$ are used in intermediate step.

The iterative relation in Beeman's method is
\begin{eqnarray}
x(t+\Delta t) &=& x(t) + v(t)\Delta t +\frac{1}{6}(4a(t)-a(t-\Delta
t)) \Delta t^2 \label{eq:beeman1}\\
v(t+\Delta t) &=& v(t) + \frac{1}{6}(2a(t+\Delta t) + 5a(t) -
a(t-\Delta t))\Delta t \label{eq:beeman2}
\end{eqnarray}
It is obviously that the expression is much more complicated than
that in Verlet's method. Nevertheless, this method is more accuracy
than Verlet's. In Beeman's algorithm, the local error is
$\mathcal{O}(\Delta t^4)$ in position and $\mathcal{O}(\Delta t^3)$
in velocity, and the global error is $\mathcal{O}(\Delta t^3)$. For
comparison, the local error is $\mathcal{O}(\Delta t^4)$ in position
and $\mathcal{O}(\Delta t^2)$ in velocity, and the global error is
$\mathcal{O}(\Delta t^2)$ in Verlet's algorithm.

Another observation from Eq.\ref{eq:beeman1} and Eq.\ref{eq:beeman2}
is that variables at both times $t$ and $t-\Delta t$ are needed to
get $t+\Delta t$ ones.

Therefore, in MOPDYN13, Beeman algorithm is used to solve the
Newtonian equations of atoms at time \mcode{t>DT}, with Verlet
algorithm as an initiation to obtain position, velocity and
acceleration at time \mcode{DT} only.



In summary, $x_0$ and $v_0$ are provided as initial condition and
one has obtained $x_1$ and $v_1$ from other algorithm such as Verlet
algorithm. Based on these, one firstly gets $a_0$ and $a_1$ from
force law such as gradient of LJ potential or QM potential by
calling MOPAC2012.exe , secondly solves $x_2$ from
Eq.\ref{eq:beeman1}, and then obtains $a_2$ from force law again,
finally solves $v_2$ from Eq.\ref{eq:beeman2}, and so forth.


\subsection{Temperature}
From the principle of equipartition of energy, translation
temperature $T$ is defined as
\begin{equation}
T=\frac{2K}{k_B N_{dof}}=\frac{1}{k_B N_{dof}}\sum^N_{i=1}m_i v_i^2
\end{equation}
where $k_B$ is Boltzmann constant and $N_{dof}$ is the number of
degrees of freedom.

In the same way, the rotational temperature $T_r$ is defined as
\[
\frac{1}{2}k_B T_r N_{dofr} = K = \frac{1}{2}\omega I \omega
\]
where $N_{dofr}$ is the number of degrees of freedom in rotation
which should be equal to $3$, $\omega$ is angular velocity, and $I$
is the moment of inertia. If we write them in tensor form
\begin{equation}
T_r = \frac{1}{3 k_B} I_{ij}\omega^i\omega^j
\end{equation}
where the Einstein summation rule is used.

(Does this kind of definition of temperature is the real temperature
in thermodynamics? Principle of equipartition of energy is based on
ensemble.)

\subsection{Temperature Control}\label{sec:tem}
There are four different approach to control temperature, which are
represented by five logical keywords
\mcode{LTEMP,LTSTAT,RTSTAT,LTRAMP} and \mcode{LQUENCH}, among which
\mcode{LTSTAT} is the most frequently used.

\subsubsection*{LTEMP}
This is very brute force method of controlling temperature. If the
temperature of the system is $T$, and the thermal bath temperature
is $T_{ref}$, in this method the velocity will be scaled by a factor
of $\sqrt{T_{ref}/T}$.

In MOPDYN13, the velocities will be rescaled by a factor of
$\sqrt{T_{ref}/T}$ every \mcode{DT*NTIME} $fs$ in order to keep the
temperature at $T_{ref}$ if \mcode{LTEMP} is true.

\subsubsection*{LTSTAT}
If \mcode{LTSTAT=T}, Berendsen(1984)'s method will be used to
thermalize the system to a temperature $T_{ref}$. In this method,
the temperature will evolve in the following manner
\begin{equation}
T_{ref}-T = (T_{ref}-T_0)\exp(-\frac{t}{\tau}) \label{eq:berendsen}
\end{equation}
where $T_0$ is the temperature at time $t=0$. As shown in
Eq.\ref{eq:berendsen}, $\tau$ is the characteristic time of
thermalization.

In MOPDYN13, the velocities will be rescaled every \mcode{DT} $fs$
by a factor of
\[\sqrt{1+\frac{1}{\textrm{TCONST}} \left[ \frac{T_{ref}}{T}-1\right]}  \]
which is the discrete version of Eq.\ref{eq:berendsen} if
\mcode{LTSTAT=T}, where $\textrm{TCONST}$ is actually
$\frac{\tau}{DT}$. As a result, the temperature after rescaling
becomes
\[ T + \frac{1}{\textrm{TCONST}}(T_{ref}-T). \] Therefore, if we assign
\mcode{TCONST=1}, it degenerates into adhoc case (\mcode{LTEMP=T});
if we assign \mcode{TCONST >> DT*NTIME*NSNAP}, there is no
temperature control at all. So \mcode{TCONST} should be assigned a
suitable value in thermalization.

\subsubsection*{RTSTAT}
It is the same as \mcode{LTSTAT}, except for rotation. It will
rescale $\vec{\omega}\times\vec{r}$, instead of $v$.

\subsubsection*{LTRAMP and LQUENCH}
They can be found in relevant part in Sec.\ref{sec:key}.

%%%%%%%%%%%%%%%%%%%%%%%%%%%%%%%%%%%%%%%%%%%%%%%%%%%%%%%%%%%%%%%%%%%%%%
%%%%%%%%%%%%%%%%%%%%%%%%%%%%%%%%%%%%%%%%%%%%%%%%%%%%%%%%%%%%%%%%%%%%%%

\section{Other Issues}

\subsection{MOPAC}\label{sec:mopac}
There are $256$ keywords in MOPAC. Among them, $12$ keywords have
been used in our example in Sec.\ref{sec:key}. They are \mcode{PM6
uhf bonds 1SCF SCFCRT=1.D-10 AUX(PRECISION=9) denout (oldens) xyz
gradients dcart geo-ok}. In the following, these $12$ keywords will
be explained firstly, and then other useful keywords will be
introduced.

\subsubsection*{Keywords in Example}
\begin{itemize}

\item \mcode{PM6}: Use the PM6 Hamiltonian in which atomic and diatomic parameters both
exist. In semi-empirical method, lots of x-electron x-integral is
parameterized.

\item \mcode{uhf}: Use the Unrestricted Hartree-Fock method.

\item \mcode{bonds}: Print final bond-order matrix, the non-diagonal element of which is defined as the sum of
the squares of the density matrix elements connecting any two atoms.
The diagonal terms are the valencies calculated from the atomic
terms only and are defined as the sum of the bonds the atom makes
with other atoms. In UHF and non-variationally optimized
wavefunctions the calculated valency will be incorrect, the degree
of error being proportional to the non-duodempotency of the total
density matrix.

\item \mcode{1SCF}: Do one scf and then stop, the same as single-point
energy calculation in Gaussian. When 1SCF is used on its own (that
is, RESTART is not also used), then derivatives will only be
calculated if GRAD is also specified.

\item \mcode{SCFCRT=1.D-10}: The default SCF criterion, $0.0001$ kcal/mol,
is to be replaced by $10^{-10}$ to make the potential energy more
accuracy. The SCF criterion is the change in energy in kcal/mol on
two successive iterations. Other minor criteria may make the
requirements for an SCF slightly more stringent. The SCF criterion
can be varied from about $1.0$ to $10^{-25}$, although numbers in
the range $0.1$ to $10^{-9}$ will suffice for most applications.

\item \mcode{AUX(PRECISION=9)}: Output auxiliary information in \mcode{mopac.aux}
for use by other programs. To increase precision by m digits (m
limited to the range 1 to 9), use PRECISION=m.

\item \mcode{denout}: Density matrix output in file
\mcode{mopac.den}.

\item \mcode{oldens}: Read initial density matrix off disk. In MD
simulation, \mcode{oldens} will be automatically added to keyword
list if MOPAC is not called for the first time.

\item \mcode{xyz}: Do all geometric operations in Cartesian
coordinates.

\item \mcode{gradients}: Print all gradients. In a 1SCF calculation gradients
are not calculated by default: in non-variationally optimized
systems this could take a lot of time. GRADIENTS allows the
gradients to be calculated. Normally, gradients will not be printed
if the gradient norm is less than $2.0$. However, if GRADIENTS is
present, then the gradient norm and the gradients will
unconditionally be printed. Abbreviation: GRAD.

\item \mcode{dcart}: The Cartesian derivatives which are calculated
in DCART for variationally optimized systems are printed if the
keyword DCART is present. The derivatives are in units of
kcals/$\AA$, and the coordinates are displacements in x, y, and z.

\item \mcode{geo-ok}: Override some safety checks. Normally the
program will stop if certain errors occur. If GEO-OK is present, the
job will continue, but there is an increased probability that the
results will not be meaningful. The errors that are affected are: If
two atoms are within $0.1\AA$ of each other. In practice, most jobs
that terminate due to these checks contain errors in data, so
caution should be exercised if GEO-OK is used.

\end{itemize}

In summary, the keywords \mcode{PM6 uhf bonds 1SCF} and
\mcode{AUX(PRECISION=9)} are always necessary, and the keywords
\mcode{denout (oldens) xyz gradients dcart geo-ok} will be added in
to keyword list by MOPDYN13 automatically. While, the keywords
\mcode{SCFCRT=1.D-10} are not necessary.

\subsubsection*{Other Useful Keywords}
\begin{itemize}
\item \mcode{ITRY}: Set limit of number of SCF iterations to n.
The default maximum number of SCF iterations is 2000. When this 
limit presents difficulty, ITRY=nn can be used to re-define it. For 
example, if ITRY=4000 is used, the maximum number of iterations will 
be set to 4000.

\item \mcode{PL}: Output the convergence informtion in mopac.output.
A tool for monitoring the behavior of SCF 
calculations is useful when they take too long, or even fail 
altogether. When keyword PL is present, the energy of the system and 
the rate of change of the electronic structure can be monitored 
iteration to iteration.  This keyword is useful particularly when 
trying out various combinations of convergers such as PULAY, KING, 
SHIFT, and DAMP to find the method that works best. The rate of 
change in the electronic structure is given by the quantity PLS in 
the output.  For example, in the line:

\begin{lstlisting}
 ITERATION 7 PLS= 0.168E-01 0.395E-06 ENERGY -34.585270 DELTAE   -2.2653063
\end{lstlisting}

the alpha wavefunction changed by 0.0168 between iteration 6 and 
iteration 7. At the same time, the beta wavefunction changed by 
0.000000395. The change in energy over these iterations is -2.265 
kcal/mol.

If the calculation uses a restricted Hartree-Fock method, then the 
line would look like this:

\begin{lstlisting}
ITERATION 7 PLS= 0.900E-02 0.000E+00 ENERGY 46.712559 DELTAE 
-0.3203785 
\end{lstlisting} 

Now the change in total wavefunction between iteration 6 and 7 would 
be 0.009.  The second number, here 0.000E+00, is not used and should 
be ignored.

\item \mcode{SHIFT}: In an attempt to obtain an SCF by damping 
oscillations which slow down the convergence or prevent an SCF being 
achieved, the virtual M.O. energy levels are shifted up or down in 
energy by a shift technique. The principle is that, if the virtual 
M.O.s are changed in energy relative to the occupied set, then the 
polarizability of the occupied M.O.s will change pro rata. Normally, 
oscillations are due to autoregenerative charge fluctuations. The 
SHIFT method has been re-written so that the value of SHIFT changes 
automatically in an attempt to optimize convergence. This can result 
in a positive or zero shift of the virtual M.O. energy levels. A 
SHIFT of 20 will raise the virtual M.O.s by 20 eV above their 
correct value. The disadvantage of SHIFT is that a large value can 
lead to excessive damping, and thus prevent an SCF being generated. 
As some virtual M.O.s are used in non-variationally optimised 
calculations SHIFT is automatically anulled at the end of the SCF in 
these circumstances. 

\item \mcode{PULAY}: Different SCF convergers. The default converger 
in the SCF calculation is to be replaced by Pulay's procedure [48] 
as soon as the density matrix is sufficiently stable.   A 
considerable improvement in speed can frequently be achieved by the 
use of PULAY, particularly for excited states. 

\item \mcode{KING}: Use Camp-King converger for SCF.The
Camp-King converger is to be used. This is a very powerful, but CPU
intensive, SCF converger.

\item \mcode{LARGE}: Print expanded output.

\item \mcode{MULLIK}: Print the Mulliken population analysis.

\item \mcode{RHF}: Use Restricted Hartree-Fock methods.
\end{itemize}

\subsubsection*{Cartesian Coordinate Definition}

A definition of geometry in Cartesian coordinates consists of the
chemical symbol or atomic number, followed by the Cartesian
coordinates and optimization flags.

Here is an example of a MOPAC data-set for formaldehyde, in which
all coordinates are Cartesian.
\begin{lstlisting}
O 0.00  0  0.00  0 0.00  0
C 1.21  1  0.00  0 0.00  0
H 1.79  1  0.93  1 0.00  0
H 1.79  1 -0.93  1 0.00  0
\end{lstlisting}
the optimization flag "1" after the "0.93" indicates that the y
coordinate of the first hydrogen atom is to be optimized, and the
optimization flay "0" indicates that the x (or y or z) coordinate of
that atom is not to be optimized.


\subsection{Unit}
The time is in $fs$. All the masses are in unit of $amu$. All the
length and coordinates are in the unit of $\AA$. All the velocities
are in unit of $\AA/fs$. All the kinetic energies are in unit of
$amu (\AA/fs)^2$, but when the kinetic energy adds with potential
energy, its unit turns to Hartrees. The potential obtained by MOPAC
is in $eV$, and changed to Hartrees afterwards. The force obtained
by MOPAC is in $kcal/mol/\AA$, and changed to $amu \AA/fs^2$ later.
The acceleration is in $\AA/fs^2$.

\subsection{Counter}
\begin{itemize}
\item \mcode{IACCEL}: the number of times of calling \mcode{subroutine accel}
\item \mcode{IRUN}: \mcode{IRUN-1 = ISNAP}
\item \mcode{IMD}: the number of times of calling \mcode{subroutine beeman}
\item \mcode{ITIME}: counter of inner loop in \mcode{subroutine dynmol}, ranging from \mcode{1} to \mcode{NTIME}
\item \mcode{ISNAP}: counter of outer loop in \mcode{subroutine
dynmol}, ranging from \mcode{1} to \mcode{NSNAP}, which is the
number of outputs.
\item \mcode{MopIteration}: the number of times of calling
MOPAC2012.exe
\end{itemize}

\section{Example*}
What will happend if LENER=F and LTSTAT=F in collision.


%%%%%%%%%%%%%%%%%%%%%%%%%%%%%%%%%%%%%%%%%%%%%%%%%%%%%%%%%%%%%%%%%%%%%%%
%%%%%%%%%%%%%%%%%%%%%%%%%%%%%%%%%%%%%%%%%%%%%%%%%%%%%%%%%%%%%%%%%%%%%%%

\acknowledgments

\appendix

\section{Appendix I}

\begin{thebibliography}{999}
\end{thebibliography}
\end{document}
